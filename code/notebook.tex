
% Default to the notebook output style

    


% Inherit from the specified cell style.




    
\documentclass[11pt]{article}

    
    
    \usepackage[T1]{fontenc}
    % Nicer default font (+ math font) than Computer Modern for most use cases
    \usepackage{mathpazo}

    % Basic figure setup, for now with no caption control since it's done
    % automatically by Pandoc (which extracts ![](path) syntax from Markdown).
    \usepackage{graphicx}
    % We will generate all images so they have a width \maxwidth. This means
    % that they will get their normal width if they fit onto the page, but
    % are scaled down if they would overflow the margins.
    \makeatletter
    \def\maxwidth{\ifdim\Gin@nat@width>\linewidth\linewidth
    \else\Gin@nat@width\fi}
    \makeatother
    \let\Oldincludegraphics\includegraphics
    % Set max figure width to be 80% of text width, for now hardcoded.
    \renewcommand{\includegraphics}[1]{\Oldincludegraphics[width=.8\maxwidth]{#1}}
    % Ensure that by default, figures have no caption (until we provide a
    % proper Figure object with a Caption API and a way to capture that
    % in the conversion process - todo).
    \usepackage{caption}
    \DeclareCaptionLabelFormat{nolabel}{}
    \captionsetup{labelformat=nolabel}

    \usepackage{adjustbox} % Used to constrain images to a maximum size 
    \usepackage{xcolor} % Allow colors to be defined
    \usepackage{enumerate} % Needed for markdown enumerations to work
    \usepackage{geometry} % Used to adjust the document margins
    \usepackage{amsmath} % Equations
    \usepackage{amssymb} % Equations
    \usepackage{textcomp} % defines textquotesingle
    % Hack from http://tex.stackexchange.com/a/47451/13684:
    \AtBeginDocument{%
        \def\PYZsq{\textquotesingle}% Upright quotes in Pygmentized code
    }
    \usepackage{upquote} % Upright quotes for verbatim code
    \usepackage{eurosym} % defines \euro
    \usepackage[mathletters]{ucs} % Extended unicode (utf-8) support
    \usepackage[utf8x]{inputenc} % Allow utf-8 characters in the tex document
    \usepackage{fancyvrb} % verbatim replacement that allows latex
    \usepackage{grffile} % extends the file name processing of package graphics 
                         % to support a larger range 
    % The hyperref package gives us a pdf with properly built
    % internal navigation ('pdf bookmarks' for the table of contents,
    % internal cross-reference links, web links for URLs, etc.)
    \usepackage{hyperref}
    \usepackage{longtable} % longtable support required by pandoc >1.10
    \usepackage{booktabs}  % table support for pandoc > 1.12.2
    \usepackage[inline]{enumitem} % IRkernel/repr support (it uses the enumerate* environment)
    \usepackage[normalem]{ulem} % ulem is needed to support strikethroughs (\sout)
                                % normalem makes italics be italics, not underlines
    

    
    
    % Colors for the hyperref package
    \definecolor{urlcolor}{rgb}{0,.145,.698}
    \definecolor{linkcolor}{rgb}{.71,0.21,0.01}
    \definecolor{citecolor}{rgb}{.12,.54,.11}

    % ANSI colors
    \definecolor{ansi-black}{HTML}{3E424D}
    \definecolor{ansi-black-intense}{HTML}{282C36}
    \definecolor{ansi-red}{HTML}{E75C58}
    \definecolor{ansi-red-intense}{HTML}{B22B31}
    \definecolor{ansi-green}{HTML}{00A250}
    \definecolor{ansi-green-intense}{HTML}{007427}
    \definecolor{ansi-yellow}{HTML}{DDB62B}
    \definecolor{ansi-yellow-intense}{HTML}{B27D12}
    \definecolor{ansi-blue}{HTML}{208FFB}
    \definecolor{ansi-blue-intense}{HTML}{0065CA}
    \definecolor{ansi-magenta}{HTML}{D160C4}
    \definecolor{ansi-magenta-intense}{HTML}{A03196}
    \definecolor{ansi-cyan}{HTML}{60C6C8}
    \definecolor{ansi-cyan-intense}{HTML}{258F8F}
    \definecolor{ansi-white}{HTML}{C5C1B4}
    \definecolor{ansi-white-intense}{HTML}{A1A6B2}

    % commands and environments needed by pandoc snippets
    % extracted from the output of `pandoc -s`
    \providecommand{\tightlist}{%
      \setlength{\itemsep}{0pt}\setlength{\parskip}{0pt}}
    \DefineVerbatimEnvironment{Highlighting}{Verbatim}{commandchars=\\\{\}}
    % Add ',fontsize=\small' for more characters per line
    \newenvironment{Shaded}{}{}
    \newcommand{\KeywordTok}[1]{\textcolor[rgb]{0.00,0.44,0.13}{\textbf{{#1}}}}
    \newcommand{\DataTypeTok}[1]{\textcolor[rgb]{0.56,0.13,0.00}{{#1}}}
    \newcommand{\DecValTok}[1]{\textcolor[rgb]{0.25,0.63,0.44}{{#1}}}
    \newcommand{\BaseNTok}[1]{\textcolor[rgb]{0.25,0.63,0.44}{{#1}}}
    \newcommand{\FloatTok}[1]{\textcolor[rgb]{0.25,0.63,0.44}{{#1}}}
    \newcommand{\CharTok}[1]{\textcolor[rgb]{0.25,0.44,0.63}{{#1}}}
    \newcommand{\StringTok}[1]{\textcolor[rgb]{0.25,0.44,0.63}{{#1}}}
    \newcommand{\CommentTok}[1]{\textcolor[rgb]{0.38,0.63,0.69}{\textit{{#1}}}}
    \newcommand{\OtherTok}[1]{\textcolor[rgb]{0.00,0.44,0.13}{{#1}}}
    \newcommand{\AlertTok}[1]{\textcolor[rgb]{1.00,0.00,0.00}{\textbf{{#1}}}}
    \newcommand{\FunctionTok}[1]{\textcolor[rgb]{0.02,0.16,0.49}{{#1}}}
    \newcommand{\RegionMarkerTok}[1]{{#1}}
    \newcommand{\ErrorTok}[1]{\textcolor[rgb]{1.00,0.00,0.00}{\textbf{{#1}}}}
    \newcommand{\NormalTok}[1]{{#1}}
    
    % Additional commands for more recent versions of Pandoc
    \newcommand{\ConstantTok}[1]{\textcolor[rgb]{0.53,0.00,0.00}{{#1}}}
    \newcommand{\SpecialCharTok}[1]{\textcolor[rgb]{0.25,0.44,0.63}{{#1}}}
    \newcommand{\VerbatimStringTok}[1]{\textcolor[rgb]{0.25,0.44,0.63}{{#1}}}
    \newcommand{\SpecialStringTok}[1]{\textcolor[rgb]{0.73,0.40,0.53}{{#1}}}
    \newcommand{\ImportTok}[1]{{#1}}
    \newcommand{\DocumentationTok}[1]{\textcolor[rgb]{0.73,0.13,0.13}{\textit{{#1}}}}
    \newcommand{\AnnotationTok}[1]{\textcolor[rgb]{0.38,0.63,0.69}{\textbf{\textit{{#1}}}}}
    \newcommand{\CommentVarTok}[1]{\textcolor[rgb]{0.38,0.63,0.69}{\textbf{\textit{{#1}}}}}
    \newcommand{\VariableTok}[1]{\textcolor[rgb]{0.10,0.09,0.49}{{#1}}}
    \newcommand{\ControlFlowTok}[1]{\textcolor[rgb]{0.00,0.44,0.13}{\textbf{{#1}}}}
    \newcommand{\OperatorTok}[1]{\textcolor[rgb]{0.40,0.40,0.40}{{#1}}}
    \newcommand{\BuiltInTok}[1]{{#1}}
    \newcommand{\ExtensionTok}[1]{{#1}}
    \newcommand{\PreprocessorTok}[1]{\textcolor[rgb]{0.74,0.48,0.00}{{#1}}}
    \newcommand{\AttributeTok}[1]{\textcolor[rgb]{0.49,0.56,0.16}{{#1}}}
    \newcommand{\InformationTok}[1]{\textcolor[rgb]{0.38,0.63,0.69}{\textbf{\textit{{#1}}}}}
    \newcommand{\WarningTok}[1]{\textcolor[rgb]{0.38,0.63,0.69}{\textbf{\textit{{#1}}}}}
    
    
    % Define a nice break command that doesn't care if a line doesn't already
    % exist.
    \def\br{\hspace*{\fill} \\* }
    % Math Jax compatability definitions
    \def\gt{>}
    \def\lt{<}
    % Document parameters
    \title{?11?-?????}
    
    
    

    % Pygments definitions
    
\makeatletter
\def\PY@reset{\let\PY@it=\relax \let\PY@bf=\relax%
    \let\PY@ul=\relax \let\PY@tc=\relax%
    \let\PY@bc=\relax \let\PY@ff=\relax}
\def\PY@tok#1{\csname PY@tok@#1\endcsname}
\def\PY@toks#1+{\ifx\relax#1\empty\else%
    \PY@tok{#1}\expandafter\PY@toks\fi}
\def\PY@do#1{\PY@bc{\PY@tc{\PY@ul{%
    \PY@it{\PY@bf{\PY@ff{#1}}}}}}}
\def\PY#1#2{\PY@reset\PY@toks#1+\relax+\PY@do{#2}}

\expandafter\def\csname PY@tok@w\endcsname{\def\PY@tc##1{\textcolor[rgb]{0.73,0.73,0.73}{##1}}}
\expandafter\def\csname PY@tok@c\endcsname{\let\PY@it=\textit\def\PY@tc##1{\textcolor[rgb]{0.25,0.50,0.50}{##1}}}
\expandafter\def\csname PY@tok@cp\endcsname{\def\PY@tc##1{\textcolor[rgb]{0.74,0.48,0.00}{##1}}}
\expandafter\def\csname PY@tok@k\endcsname{\let\PY@bf=\textbf\def\PY@tc##1{\textcolor[rgb]{0.00,0.50,0.00}{##1}}}
\expandafter\def\csname PY@tok@kp\endcsname{\def\PY@tc##1{\textcolor[rgb]{0.00,0.50,0.00}{##1}}}
\expandafter\def\csname PY@tok@kt\endcsname{\def\PY@tc##1{\textcolor[rgb]{0.69,0.00,0.25}{##1}}}
\expandafter\def\csname PY@tok@o\endcsname{\def\PY@tc##1{\textcolor[rgb]{0.40,0.40,0.40}{##1}}}
\expandafter\def\csname PY@tok@ow\endcsname{\let\PY@bf=\textbf\def\PY@tc##1{\textcolor[rgb]{0.67,0.13,1.00}{##1}}}
\expandafter\def\csname PY@tok@nb\endcsname{\def\PY@tc##1{\textcolor[rgb]{0.00,0.50,0.00}{##1}}}
\expandafter\def\csname PY@tok@nf\endcsname{\def\PY@tc##1{\textcolor[rgb]{0.00,0.00,1.00}{##1}}}
\expandafter\def\csname PY@tok@nc\endcsname{\let\PY@bf=\textbf\def\PY@tc##1{\textcolor[rgb]{0.00,0.00,1.00}{##1}}}
\expandafter\def\csname PY@tok@nn\endcsname{\let\PY@bf=\textbf\def\PY@tc##1{\textcolor[rgb]{0.00,0.00,1.00}{##1}}}
\expandafter\def\csname PY@tok@ne\endcsname{\let\PY@bf=\textbf\def\PY@tc##1{\textcolor[rgb]{0.82,0.25,0.23}{##1}}}
\expandafter\def\csname PY@tok@nv\endcsname{\def\PY@tc##1{\textcolor[rgb]{0.10,0.09,0.49}{##1}}}
\expandafter\def\csname PY@tok@no\endcsname{\def\PY@tc##1{\textcolor[rgb]{0.53,0.00,0.00}{##1}}}
\expandafter\def\csname PY@tok@nl\endcsname{\def\PY@tc##1{\textcolor[rgb]{0.63,0.63,0.00}{##1}}}
\expandafter\def\csname PY@tok@ni\endcsname{\let\PY@bf=\textbf\def\PY@tc##1{\textcolor[rgb]{0.60,0.60,0.60}{##1}}}
\expandafter\def\csname PY@tok@na\endcsname{\def\PY@tc##1{\textcolor[rgb]{0.49,0.56,0.16}{##1}}}
\expandafter\def\csname PY@tok@nt\endcsname{\let\PY@bf=\textbf\def\PY@tc##1{\textcolor[rgb]{0.00,0.50,0.00}{##1}}}
\expandafter\def\csname PY@tok@nd\endcsname{\def\PY@tc##1{\textcolor[rgb]{0.67,0.13,1.00}{##1}}}
\expandafter\def\csname PY@tok@s\endcsname{\def\PY@tc##1{\textcolor[rgb]{0.73,0.13,0.13}{##1}}}
\expandafter\def\csname PY@tok@sd\endcsname{\let\PY@it=\textit\def\PY@tc##1{\textcolor[rgb]{0.73,0.13,0.13}{##1}}}
\expandafter\def\csname PY@tok@si\endcsname{\let\PY@bf=\textbf\def\PY@tc##1{\textcolor[rgb]{0.73,0.40,0.53}{##1}}}
\expandafter\def\csname PY@tok@se\endcsname{\let\PY@bf=\textbf\def\PY@tc##1{\textcolor[rgb]{0.73,0.40,0.13}{##1}}}
\expandafter\def\csname PY@tok@sr\endcsname{\def\PY@tc##1{\textcolor[rgb]{0.73,0.40,0.53}{##1}}}
\expandafter\def\csname PY@tok@ss\endcsname{\def\PY@tc##1{\textcolor[rgb]{0.10,0.09,0.49}{##1}}}
\expandafter\def\csname PY@tok@sx\endcsname{\def\PY@tc##1{\textcolor[rgb]{0.00,0.50,0.00}{##1}}}
\expandafter\def\csname PY@tok@m\endcsname{\def\PY@tc##1{\textcolor[rgb]{0.40,0.40,0.40}{##1}}}
\expandafter\def\csname PY@tok@gh\endcsname{\let\PY@bf=\textbf\def\PY@tc##1{\textcolor[rgb]{0.00,0.00,0.50}{##1}}}
\expandafter\def\csname PY@tok@gu\endcsname{\let\PY@bf=\textbf\def\PY@tc##1{\textcolor[rgb]{0.50,0.00,0.50}{##1}}}
\expandafter\def\csname PY@tok@gd\endcsname{\def\PY@tc##1{\textcolor[rgb]{0.63,0.00,0.00}{##1}}}
\expandafter\def\csname PY@tok@gi\endcsname{\def\PY@tc##1{\textcolor[rgb]{0.00,0.63,0.00}{##1}}}
\expandafter\def\csname PY@tok@gr\endcsname{\def\PY@tc##1{\textcolor[rgb]{1.00,0.00,0.00}{##1}}}
\expandafter\def\csname PY@tok@ge\endcsname{\let\PY@it=\textit}
\expandafter\def\csname PY@tok@gs\endcsname{\let\PY@bf=\textbf}
\expandafter\def\csname PY@tok@gp\endcsname{\let\PY@bf=\textbf\def\PY@tc##1{\textcolor[rgb]{0.00,0.00,0.50}{##1}}}
\expandafter\def\csname PY@tok@go\endcsname{\def\PY@tc##1{\textcolor[rgb]{0.53,0.53,0.53}{##1}}}
\expandafter\def\csname PY@tok@gt\endcsname{\def\PY@tc##1{\textcolor[rgb]{0.00,0.27,0.87}{##1}}}
\expandafter\def\csname PY@tok@err\endcsname{\def\PY@bc##1{\setlength{\fboxsep}{0pt}\fcolorbox[rgb]{1.00,0.00,0.00}{1,1,1}{\strut ##1}}}
\expandafter\def\csname PY@tok@kc\endcsname{\let\PY@bf=\textbf\def\PY@tc##1{\textcolor[rgb]{0.00,0.50,0.00}{##1}}}
\expandafter\def\csname PY@tok@kd\endcsname{\let\PY@bf=\textbf\def\PY@tc##1{\textcolor[rgb]{0.00,0.50,0.00}{##1}}}
\expandafter\def\csname PY@tok@kn\endcsname{\let\PY@bf=\textbf\def\PY@tc##1{\textcolor[rgb]{0.00,0.50,0.00}{##1}}}
\expandafter\def\csname PY@tok@kr\endcsname{\let\PY@bf=\textbf\def\PY@tc##1{\textcolor[rgb]{0.00,0.50,0.00}{##1}}}
\expandafter\def\csname PY@tok@bp\endcsname{\def\PY@tc##1{\textcolor[rgb]{0.00,0.50,0.00}{##1}}}
\expandafter\def\csname PY@tok@fm\endcsname{\def\PY@tc##1{\textcolor[rgb]{0.00,0.00,1.00}{##1}}}
\expandafter\def\csname PY@tok@vc\endcsname{\def\PY@tc##1{\textcolor[rgb]{0.10,0.09,0.49}{##1}}}
\expandafter\def\csname PY@tok@vg\endcsname{\def\PY@tc##1{\textcolor[rgb]{0.10,0.09,0.49}{##1}}}
\expandafter\def\csname PY@tok@vi\endcsname{\def\PY@tc##1{\textcolor[rgb]{0.10,0.09,0.49}{##1}}}
\expandafter\def\csname PY@tok@vm\endcsname{\def\PY@tc##1{\textcolor[rgb]{0.10,0.09,0.49}{##1}}}
\expandafter\def\csname PY@tok@sa\endcsname{\def\PY@tc##1{\textcolor[rgb]{0.73,0.13,0.13}{##1}}}
\expandafter\def\csname PY@tok@sb\endcsname{\def\PY@tc##1{\textcolor[rgb]{0.73,0.13,0.13}{##1}}}
\expandafter\def\csname PY@tok@sc\endcsname{\def\PY@tc##1{\textcolor[rgb]{0.73,0.13,0.13}{##1}}}
\expandafter\def\csname PY@tok@dl\endcsname{\def\PY@tc##1{\textcolor[rgb]{0.73,0.13,0.13}{##1}}}
\expandafter\def\csname PY@tok@s2\endcsname{\def\PY@tc##1{\textcolor[rgb]{0.73,0.13,0.13}{##1}}}
\expandafter\def\csname PY@tok@sh\endcsname{\def\PY@tc##1{\textcolor[rgb]{0.73,0.13,0.13}{##1}}}
\expandafter\def\csname PY@tok@s1\endcsname{\def\PY@tc##1{\textcolor[rgb]{0.73,0.13,0.13}{##1}}}
\expandafter\def\csname PY@tok@mb\endcsname{\def\PY@tc##1{\textcolor[rgb]{0.40,0.40,0.40}{##1}}}
\expandafter\def\csname PY@tok@mf\endcsname{\def\PY@tc##1{\textcolor[rgb]{0.40,0.40,0.40}{##1}}}
\expandafter\def\csname PY@tok@mh\endcsname{\def\PY@tc##1{\textcolor[rgb]{0.40,0.40,0.40}{##1}}}
\expandafter\def\csname PY@tok@mi\endcsname{\def\PY@tc##1{\textcolor[rgb]{0.40,0.40,0.40}{##1}}}
\expandafter\def\csname PY@tok@il\endcsname{\def\PY@tc##1{\textcolor[rgb]{0.40,0.40,0.40}{##1}}}
\expandafter\def\csname PY@tok@mo\endcsname{\def\PY@tc##1{\textcolor[rgb]{0.40,0.40,0.40}{##1}}}
\expandafter\def\csname PY@tok@ch\endcsname{\let\PY@it=\textit\def\PY@tc##1{\textcolor[rgb]{0.25,0.50,0.50}{##1}}}
\expandafter\def\csname PY@tok@cm\endcsname{\let\PY@it=\textit\def\PY@tc##1{\textcolor[rgb]{0.25,0.50,0.50}{##1}}}
\expandafter\def\csname PY@tok@cpf\endcsname{\let\PY@it=\textit\def\PY@tc##1{\textcolor[rgb]{0.25,0.50,0.50}{##1}}}
\expandafter\def\csname PY@tok@c1\endcsname{\let\PY@it=\textit\def\PY@tc##1{\textcolor[rgb]{0.25,0.50,0.50}{##1}}}
\expandafter\def\csname PY@tok@cs\endcsname{\let\PY@it=\textit\def\PY@tc##1{\textcolor[rgb]{0.25,0.50,0.50}{##1}}}

\def\PYZbs{\char`\\}
\def\PYZus{\char`\_}
\def\PYZob{\char`\{}
\def\PYZcb{\char`\}}
\def\PYZca{\char`\^}
\def\PYZam{\char`\&}
\def\PYZlt{\char`\<}
\def\PYZgt{\char`\>}
\def\PYZsh{\char`\#}
\def\PYZpc{\char`\%}
\def\PYZdl{\char`\$}
\def\PYZhy{\char`\-}
\def\PYZsq{\char`\'}
\def\PYZdq{\char`\"}
\def\PYZti{\char`\~}
% for compatibility with earlier versions
\def\PYZat{@}
\def\PYZlb{[}
\def\PYZrb{]}
\makeatother


    % Exact colors from NB
    \definecolor{incolor}{rgb}{0.0, 0.0, 0.5}
    \definecolor{outcolor}{rgb}{0.545, 0.0, 0.0}



    
    % Prevent overflowing lines due to hard-to-break entities
    \sloppy 
    % Setup hyperref package
    \hypersetup{
      breaklinks=true,  % so long urls are correctly broken across lines
      colorlinks=true,
      urlcolor=urlcolor,
      linkcolor=linkcolor,
      citecolor=citecolor,
      }
    % Slightly bigger margins than the latex defaults
    
    \geometry{verbose,tmargin=1in,bmargin=1in,lmargin=1in,rmargin=1in}
    
    

    \begin{document}
    
    
    \maketitle
    
    

    
    \subsubsection{本课提纲}\label{ux672cux8bfeux63d0ux7eb2}

\begin{itemize}
\tightlist
\item
  什么是最优化
\item
  梯度下降方法
\item
  遗传算法
\item
  练习
\end{itemize}

    \subsubsection{11.1
什么是最优化}\label{ux4ec0ux4e48ux662fux6700ux4f18ux5316}

    简单来讲,最优化就是找出函数的极值。假设有一个普通的二次方函数,例如
\(y=x^2\),我们希望知道它的最小值在哪里。从下面的图可以看到,当x取0的时候,这个函数取值最小。更形象的解释就是,如果把这个函数看作是一座山,有一个生活在这座山上的小兔子,小兔子希望能跑到山的最低洼地方去,但是它的视力又不好,一眼看不到最低处,只能看到眼前的东西,那有什么办法可以到达最低处呢?

    \begin{Verbatim}[commandchars=\\\{\}]
{\color{incolor}In [{\color{incolor}5}]:} \PY{k+kn}{import} \PY{n+nn}{numpy} \PY{k}{as} \PY{n+nn}{np}
        \PY{k+kn}{import} \PY{n+nn}{matplotlib}\PY{n+nn}{.}\PY{n+nn}{pyplot} \PY{k}{as} \PY{n+nn}{plt}
        \PY{o}{\PYZpc{}}\PY{k}{matplotlib} inline
\end{Verbatim}


    \begin{Verbatim}[commandchars=\\\{\}]
{\color{incolor}In [{\color{incolor}35}]:} \PY{n}{x} \PY{o}{=} \PY{n}{np}\PY{o}{.}\PY{n}{linspace}\PY{p}{(}\PY{o}{\PYZhy{}}\PY{l+m+mi}{2}\PY{p}{,}\PY{l+m+mi}{2}\PY{p}{,}\PY{l+m+mi}{100}\PY{p}{)}
         \PY{n}{y} \PY{o}{=} \PY{n}{x}\PY{o}{*}\PY{o}{*}\PY{l+m+mi}{2}
         \PY{n}{plt}\PY{o}{.}\PY{n}{plot}\PY{p}{(}\PY{n}{x}\PY{p}{,}\PY{n}{y}\PY{p}{)}\PY{p}{;}
         \PY{n}{plt}\PY{o}{.}\PY{n}{scatter}\PY{p}{(}\PY{l+m+mi}{0}\PY{p}{,}\PY{l+m+mi}{0}\PY{p}{,}\PY{n}{color}\PY{o}{=}\PY{l+s+s1}{\PYZsq{}}\PY{l+s+s1}{red}\PY{l+s+s1}{\PYZsq{}}\PY{p}{)}\PY{p}{;}
\end{Verbatim}


    \begin{center}
    \adjustimage{max size={0.9\linewidth}{0.9\paperheight}}{output_4_0.png}
    \end{center}
    { \hspace*{\fill} \\}
    
    \subsubsection{11.2 梯度下降}\label{ux68afux5ea6ux4e0bux964d}

    如果小兔子看不到远处,但是可以看到近处,那么它会知道在它现在所处的这个地方,地势的高低走向是怎么样的。因为它生活在这个二维的山上,它只有两个方向可以选择和观察,向左走或是向右走。它可以观察到近处左边的地势和右边的地势。可以这样来思考一下,如果左边的地势比右边的地势低,那么有可能低洼处就在左边某个地方,所以从贪心算法的思路来看,我们就以当前位置的局部地势来判断,往低处走。往左走了一步之后,在以当前位置观察局部的地势,再进行判断决策。就这样循环往复,最终会找到山的最低处。

    如果要编程来模仿小兔子的行动决策,我们还需要明确两个具体的东西,一是如何来判断局部的地势以决定走的方向,二是如何决定走的步子大小。对于第一个问题,我们可以联想到斜率这个概念,如果在山的某一个位置上,或者说在函数的某个点的位置上,这个点对应的斜率是可以算出来的,如果斜率是正数,说明左边比右边低,方向选择上就应该往左走,反之,如果斜率是负数,说明左边比右边高,方向选择上应该往右走。斜率在数学上也称之为导数或者梯度。对于第二个问题,可以想像的到,如果步子小,我们需要更多的步数才走到底部,如果步子大,可能需要较少的步数,但是呢,步子越大,可能会越过底部,走过头了。所以通常我们会设置一个谨慎的步长。

    如果用一个公式来描述上述的思路就是: \(x = x - rate\times gx\)

x是当前位置,rate是步长,gx是当前位置的斜率,在每一步,我们会算出当前位置的斜率gx,当gx为正也就是斜率为正,我们应该往左走,自然就是在当前位置上做减法,x就会变小;如果gx为负也就是斜率为负,和前面那个负号抵消后成为一个加法,x就会增加,就是往右走。

    \begin{Verbatim}[commandchars=\\\{\}]
{\color{incolor}In [{\color{incolor}66}]:} \PY{k}{def} \PY{n+nf}{min\PYZus{}gred}\PY{p}{(}\PY{n}{x\PYZus{}start}\PY{p}{,} \PY{n}{rate}\PY{p}{,} \PY{n}{num}\PY{p}{,} \PY{n}{f}\PY{p}{,}\PY{n}{g}\PY{p}{)}\PY{p}{:}
             \PY{n}{x} \PY{o}{=} \PY{n}{x\PYZus{}start}
             \PY{k}{for} \PY{n}{n} \PY{o+ow}{in} \PY{n+nb}{range}\PY{p}{(}\PY{n}{num}\PY{p}{)}\PY{p}{:}
                 \PY{n}{gx} \PY{o}{=} \PY{n}{g}\PY{p}{(}\PY{n}{x}\PY{p}{)}
                 \PY{n}{y} \PY{o}{=} \PY{n}{f}\PY{p}{(}\PY{n}{x}\PY{p}{)}
                 \PY{n}{x} \PY{o}{=} \PY{n}{x} \PY{o}{\PYZhy{}} \PY{n}{rate}\PY{o}{*}\PY{n}{gx} \PY{c+c1}{\PYZsh{} 梯度下降}
                 \PY{n+nb}{print}\PY{p}{(}\PY{l+s+s2}{\PYZdq{}}\PY{l+s+s2}{X:}\PY{l+s+si}{\PYZob{}x:.2f\PYZcb{}}\PY{l+s+s2}{, Y:}\PY{l+s+si}{\PYZob{}y:.2f\PYZcb{}}\PY{l+s+s2}{,gx:}\PY{l+s+si}{\PYZob{}gx:.2f\PYZcb{}}\PY{l+s+s2}{\PYZdq{}}\PY{o}{.}\PY{n}{format}\PY{p}{(}\PY{n}{x}\PY{o}{=}\PY{n}{x}\PY{p}{,} \PY{n}{y}\PY{o}{=}\PY{n}{y}\PY{p}{,}\PY{n}{gx}\PY{o}{=}\PY{n}{gx}\PY{p}{)}\PY{p}{)}
                 \PY{k}{if} \PY{n+nb}{abs}\PY{p}{(}\PY{n}{gx}\PY{p}{)}\PY{o}{\PYZlt{}}\PY{l+m+mf}{0.0001}\PY{p}{:}
                     \PY{k}{break}
             \PY{k}{return} \PY{n}{x}
                 
\end{Verbatim}


    下面我们来解释一下这个代码的含义,这个函数代码是上述的梯度下降思路来求某个函数的最小值。输入参数有四个,x\_start表示初始点的位置,也就是小兔子一开始的位置,rate是步长设置,num是走的步数,f是需要求最小值的那个函数,也就是山的形状,g是f的梯度函数,也就是对应山的每个位置的斜率可以通过g算出来。

    在进入循环之后,每一步,我们会算出当前位置的斜率,存在gx中,再算出当前的高度y,再用一个梯度下降公式去修正x的值,因为gx为正也就是斜率为正,我们应该往左走,自然就是在当前位置上做减法,如果gx为负也就是斜率为负,和前面那个负号抵消后成为一个加法,x就会增加,就是往右走。

    同时为了显示每一步的信息,我们打印出这几个变量的值,以方便我们调试和理解,最后有一个条件判断,当斜率非常小的时候,说明地势已经很平了,可能已经到最低点了,可以提前终止。

    \begin{Verbatim}[commandchars=\\\{\}]
{\color{incolor}In [{\color{incolor}59}]:} \PY{n}{f} \PY{o}{=} \PY{k}{lambda} \PY{n}{x}\PY{p}{:}\PY{n}{x}\PY{o}{*}\PY{o}{*}\PY{l+m+mi}{2}
         \PY{n}{g} \PY{o}{=} \PY{k}{lambda} \PY{n}{x}\PY{p}{:} \PY{l+m+mi}{2}\PY{o}{*}\PY{n}{x}
\end{Verbatim}


    为了计算平方函数的最小值,我们需要定义这个二次方函数,存放在f中,同时定义好二次方函数对应的梯度函数。梯度函数g就是对f的求导结果。

    \begin{Verbatim}[commandchars=\\\{\}]
{\color{incolor}In [{\color{incolor}65}]:} \PY{n}{min\PYZus{}gred}\PY{p}{(}\PY{l+m+mi}{2}\PY{p}{,}\PY{l+m+mf}{0.1}\PY{p}{,} \PY{l+m+mi}{10}\PY{p}{,}\PY{n}{f}\PY{p}{,}\PY{n}{g}\PY{p}{)}
\end{Verbatim}


    \begin{Verbatim}[commandchars=\\\{\}]
X:1.60, Y:4.00,gx:4.00
X:1.28, Y:2.56,gx:3.20
X:1.02, Y:1.64,gx:2.56
X:0.82, Y:1.05,gx:2.05
X:0.66, Y:0.67,gx:1.64
X:0.52, Y:0.43,gx:1.31
X:0.42, Y:0.27,gx:1.05
X:0.34, Y:0.18,gx:0.84
X:0.27, Y:0.11,gx:0.67
X:0.21, Y:0.07,gx:0.54

    \end{Verbatim}

\begin{Verbatim}[commandchars=\\\{\}]
{\color{outcolor}Out[{\color{outcolor}65}]:} 0.21474836480000006
\end{Verbatim}
            
    我们来运行这个代码,如果初始点为2,步长为0.1,走10步,我们会发现由于每一步走的比较小,最后我们走到了x=0.21这个位置,此时比较靠近0了,但是并未达到最小值。不过从打印的信息中可以看到,x是慢慢靠近0的,而且y也是在减小的,说明走的方向是对的,只不过还没走到。这时我们可以调大步数,或是调大步长。

    \begin{Verbatim}[commandchars=\\\{\}]
{\color{incolor}In [{\color{incolor}71}]:} \PY{n}{min\PYZus{}gred}\PY{p}{(}\PY{l+m+mi}{2}\PY{p}{,}\PY{l+m+mf}{0.3}\PY{p}{,} \PY{l+m+mi}{10}\PY{p}{,}\PY{n}{f}\PY{p}{,}\PY{n}{g}\PY{p}{)}
\end{Verbatim}


    \begin{Verbatim}[commandchars=\\\{\}]
X:0.80, Y:4.00,gx:4.00
X:0.32, Y:0.64,gx:1.60
X:0.13, Y:0.10,gx:0.64
X:0.05, Y:0.02,gx:0.26
X:0.02, Y:0.00,gx:0.10
X:0.01, Y:0.00,gx:0.04
X:0.00, Y:0.00,gx:0.02
X:0.00, Y:0.00,gx:0.01
X:0.00, Y:0.00,gx:0.00
X:0.00, Y:0.00,gx:0.00

    \end{Verbatim}

\begin{Verbatim}[commandchars=\\\{\}]
{\color{outcolor}Out[{\color{outcolor}71}]:} 0.00020971520000000014
\end{Verbatim}
            
    将步长调整到0.3之后,发现x已经非常接近0了,目标达到了。

    \begin{Verbatim}[commandchars=\\\{\}]
{\color{incolor}In [{\color{incolor}81}]:} \PY{n}{min\PYZus{}gred}\PY{p}{(}\PY{l+m+mi}{2}\PY{p}{,}\PY{l+m+mf}{1.1}\PY{p}{,} \PY{l+m+mi}{10}\PY{p}{,}\PY{n}{f}\PY{p}{,}\PY{n}{g}\PY{p}{)}
\end{Verbatim}


    \begin{Verbatim}[commandchars=\\\{\}]
X:-2.40, Y:4.00,gx:4.00
X:2.88, Y:5.76,gx:-4.80
X:-3.46, Y:8.29,gx:5.76
X:4.15, Y:11.94,gx:-6.91
X:-4.98, Y:17.20,gx:8.29
X:5.97, Y:24.77,gx:-9.95
X:-7.17, Y:35.66,gx:11.94
X:8.60, Y:51.36,gx:-14.33
X:-10.32, Y:73.95,gx:17.20
X:12.38, Y:106.49,gx:-20.64

    \end{Verbatim}

\begin{Verbatim}[commandchars=\\\{\}]
{\color{outcolor}Out[{\color{outcolor}81}]:} 12.383472844800014
\end{Verbatim}
            
    当我们将步长调整到1.1之后,发现不仅没有接近最小值,反而越走越远。这是因为步长设置的过大,出现震荡现象。

    小结:梯度下降是一种寻找函数极值的思路,它需要知道三个因素后进行循环计算,以逼近极值,这三个因素分别是初始的位置,函数的导数,步长。如果步长设置的过小,会花费很多步数才会到达极值,如果步长设置过大,会导致无法到达极值位置。

    \subsubsection{11.3 遗传算法}\label{ux9057ux4f20ux7b97ux6cd5}

    遗传算法是受到大自然规律的启发,如果一个生物种群要在自然界生存下去,它必须将优秀的个体遗传到下一代。以长颈鹿这个种群为例,因为自然竞争,低处的植物被很多动物吃掉了,为了生存,它们的身高需要越来越高,否则无法吃到高处的食物。上一代的长颈鹿产生下一代时,下一代的身高不会都能够到高处的食物,那些身高不足的下一代必然容易夭折早亡,那些有足够身高的下一代自然容易存活,并给再下一代传续它们的身高基因。

    从自然界的规律出发,我们可以将二次方函数想像成一个生态环境,只有在低洼处生活的小兔子才有水喝,才能够生存下来。假设一开始有一群小兔子,随机分散在二次方函数那座山上,不同高度的动物得到的水不同,越低处水资源越丰富,存活度越高,高处的动物不容易留存,而低处的动物容易存活从而留下它们的基因,低处动物的下一代会更容易往低处走,从而找到最低点。

    为了后面编程方便,我们需要定义相关的术语。

\begin{itemize}
\tightlist
\item
  种群:在生态环境中是一群生物,在二次方函数中是一组X
\item
  个体:在生态环境中是单个生物,在二次方函数中是一个X
\item
  杂交:在生态环境中是两个上一代的生物产生下一代,在二次方函数中是两个X进行加权求合
\item
  变异:在生态环境中是某个下一代的生物特征产生突变,在二次方函数中是某个X进行随机加减
\item
  适应度:在生态环境中是生物个体或种群对于环境的存活性,在二次方函数中是X对应的高度,我们希望找到最低点,所以越低越好。
\item
  父代和子代:在生态环境中是上一代和下一代,在二次方函数中迭代的前一次和后一次两组X
\end{itemize}

    首先我们初始化种群,产生最初一代的X,包括10个个体,分散在-2到+2之间的区间上。

    \begin{Verbatim}[commandchars=\\\{\}]
{\color{incolor}In [{\color{incolor}119}]:} \PY{n}{x} \PY{o}{=} \PY{n}{np}\PY{o}{.}\PY{n}{linspace}\PY{p}{(}\PY{o}{\PYZhy{}}\PY{l+m+mi}{2}\PY{p}{,}\PY{l+m+mi}{2}\PY{p}{,}\PY{l+m+mi}{100}\PY{p}{)}
          \PY{k}{def} \PY{n+nf}{func}\PY{p}{(}\PY{n}{x}\PY{p}{)}\PY{p}{:}
              \PY{k}{return} \PY{n}{x}\PY{o}{*}\PY{o}{*}\PY{l+m+mi}{2}
          \PY{n}{y} \PY{o}{=} \PY{n}{func}\PY{p}{(}\PY{n}{x}\PY{p}{)}
\end{Verbatim}


    \begin{Verbatim}[commandchars=\\\{\}]
{\color{incolor}In [{\color{incolor}140}]:} \PY{n}{num} \PY{o}{=} \PY{l+m+mi}{10}
          \PY{n}{pop} \PY{o}{=} \PY{n}{np}\PY{o}{.}\PY{n}{random}\PY{o}{.}\PY{n}{uniform}\PY{p}{(}\PY{n}{low}\PY{o}{=}\PY{o}{\PYZhy{}}\PY{l+m+mi}{2}\PY{p}{,}\PY{n}{high}\PY{o}{=}\PY{l+m+mi}{2}\PY{p}{,}\PY{n}{size}\PY{o}{=}\PY{n}{num}\PY{p}{)}
\end{Verbatim}


    \begin{Verbatim}[commandchars=\\\{\}]
{\color{incolor}In [{\color{incolor}141}]:} \PY{n}{plt}\PY{o}{.}\PY{n}{plot}\PY{p}{(}\PY{n}{x}\PY{p}{,}\PY{n}{y}\PY{p}{)}\PY{p}{;}
          \PY{n}{y\PYZus{}fit} \PY{o}{=} \PY{p}{[}\PY{n}{func}\PY{p}{(}\PY{n}{i}\PY{p}{)} \PY{k}{for} \PY{n}{i} \PY{o+ow}{in} \PY{n}{pop}\PY{p}{]}
          \PY{n}{plt}\PY{o}{.}\PY{n}{scatter}\PY{p}{(}\PY{n}{pop}\PY{p}{,} \PY{n}{y\PYZus{}fit}\PY{p}{,}\PY{n}{c}\PY{o}{=}\PY{l+s+s1}{\PYZsq{}}\PY{l+s+s1}{r}\PY{l+s+s1}{\PYZsq{}}\PY{p}{)}\PY{p}{;}
\end{Verbatim}


    \begin{center}
    \adjustimage{max size={0.9\linewidth}{0.9\paperheight}}{output_29_0.png}
    \end{center}
    { \hspace*{\fill} \\}
    
    定义适应度,用于评估个体和种群的质量,因为我们希望找到最低点,所以用二次方函数加负号来表示,这样一来,越在山高处的点,适应度值越小,越在山低处的点,适应度值越大,质量越好。此外,如果对于-2和+2以外的点,设置为一个比较大的负数,以表示质量很差。

    \begin{Verbatim}[commandchars=\\\{\}]
{\color{incolor}In [{\color{incolor}121}]:} \PY{k}{def} \PY{n+nf}{fitness}\PY{p}{(}\PY{n}{x}\PY{p}{)}\PY{p}{:}
              \PY{k}{if} \PY{p}{(}\PY{n}{x}\PY{o}{\PYZgt{}}\PY{l+m+mi}{2} \PY{o+ow}{or} \PY{n}{x}\PY{o}{\PYZlt{}}\PY{o}{\PYZhy{}}\PY{l+m+mi}{2}\PY{p}{)}\PY{p}{:}  
                  \PY{k}{return} \PY{o}{\PYZhy{}}\PY{l+m+mi}{100}
              \PY{k}{else}\PY{p}{:} 
                  \PY{k}{return} \PY{o}{\PYZhy{}}\PY{n}{x}\PY{o}{*}\PY{o}{*}\PY{l+m+mi}{2}
\end{Verbatim}


    利用前面的函数计算出整体种群的质量

    \begin{Verbatim}[commandchars=\\\{\}]
{\color{incolor}In [{\color{incolor}122}]:} \PY{k}{def} \PY{n+nf}{fitness\PYZus{}pop}\PY{p}{(}\PY{n}{pop}\PY{p}{)}\PY{p}{:}
              \PY{k}{return} \PY{n}{np}\PY{o}{.}\PY{n}{sum}\PY{p}{(}\PY{n}{fitness}\PY{p}{(}\PY{n}{x}\PY{p}{)} \PY{k}{for} \PY{n}{x} \PY{o+ow}{in} \PY{n}{pop}\PY{p}{)}\PY{o}{/}\PY{n+nb}{len}\PY{p}{(}\PY{n}{pop}\PY{p}{)}
          
          \PY{n}{fitness\PYZus{}pop}\PY{p}{(}\PY{n}{pop}\PY{p}{)}
\end{Verbatim}


\begin{Verbatim}[commandchars=\\\{\}]
{\color{outcolor}Out[{\color{outcolor}122}]:} -1.7311609764341587
\end{Verbatim}
            
    质量高的优秀个体更容易存活和有下一代,为了体现这一点,我们将适应度转换为一个选择概率,以方便后续计算。

    \begin{Verbatim}[commandchars=\\\{\}]
{\color{incolor}In [{\color{incolor}124}]:} \PY{k}{def} \PY{n+nf}{fitness\PYZus{}prob}\PY{p}{(}\PY{n}{pop}\PY{p}{)}\PY{p}{:}
              \PY{k}{return} \PY{n}{np}\PY{o}{.}\PY{n}{exp}\PY{p}{(}\PY{p}{[}\PY{n}{fitness}\PY{p}{(}\PY{n}{x}\PY{p}{)} \PY{k}{for} \PY{n}{x} \PY{o+ow}{in} \PY{n}{pop}\PY{p}{]}\PY{p}{)}\PY{o}{/}\PY{n}{np}\PY{o}{.}\PY{n}{sum}\PY{p}{(}\PY{n}{np}\PY{o}{.}\PY{n}{exp}\PY{p}{(}\PY{p}{[}\PY{n}{fitness}\PY{p}{(}\PY{n}{x}\PY{p}{)} \PY{k}{for} \PY{n}{x} \PY{o+ow}{in} \PY{n}{pop}\PY{p}{]}\PY{p}{)}\PY{p}{)}
          
          \PY{n}{fitness\PYZus{}prob}\PY{p}{(}\PY{n}{pop}\PY{p}{)}
\end{Verbatim}


\begin{Verbatim}[commandchars=\\\{\}]
{\color{outcolor}Out[{\color{outcolor}124}]:} array([0.02055009, 0.0127873 , 0.03558897, 0.30580228, 0.17240482,
                 0.13704312, 0.013488  , 0.18353671, 0.08567397, 0.03312475])
\end{Verbatim}
            
    定义杂交函数,对父代双亲进行加权求和,权重是一个随机值。

    \begin{Verbatim}[commandchars=\\\{\}]
{\color{incolor}In [{\color{incolor}126}]:} \PY{k}{def} \PY{n+nf}{crossover}\PY{p}{(}\PY{n}{p1}\PY{p}{,}\PY{n}{p2}\PY{p}{)}\PY{p}{:}
              \PY{n}{weight} \PY{o}{=} \PY{n}{np}\PY{o}{.}\PY{n}{random}\PY{o}{.}\PY{n}{rand}\PY{p}{(}\PY{p}{)}
              \PY{n}{p1\PYZus{}sub} \PY{o}{=} \PY{n}{p1}\PY{o}{*}\PY{n}{weight}
              \PY{n}{p2\PYZus{}sub} \PY{o}{=} \PY{n}{p2}\PY{o}{*}\PY{p}{(}\PY{l+m+mi}{1}\PY{o}{\PYZhy{}}\PY{n}{weight}\PY{p}{)}
              \PY{n}{children} \PY{o}{=} \PY{n}{p1\PYZus{}sub}\PY{o}{+}\PY{n}{p2\PYZus{}sub}
              \PY{k}{return} \PY{n}{children}
\end{Verbatim}


    定义变异函数,对某个X进行随机的加减,相当于对局部周围进行了随机搜索。

    \begin{Verbatim}[commandchars=\\\{\}]
{\color{incolor}In [{\color{incolor}127}]:} \PY{k}{def} \PY{n+nf}{mutate}\PY{p}{(}\PY{n}{p}\PY{p}{)}\PY{p}{:}
              \PY{n}{step} \PY{o}{=} \PY{n}{np}\PY{o}{.}\PY{n}{random}\PY{o}{.}\PY{n}{randn}\PY{p}{(}\PY{p}{)}\PY{o}{*}\PY{l+m+mf}{0.1}
              \PY{n}{children} \PY{o}{=} \PY{n}{p}\PY{o}{+}\PY{n}{step}
              \PY{k}{return} \PY{n}{children}
\end{Verbatim}


    设置一些进化参数,有多少比例的个体会参与杂交,有多少会参与变异

    \begin{Verbatim}[commandchars=\\\{\}]
{\color{incolor}In [{\color{incolor}128}]:} \PY{c+c1}{\PYZsh{} 进化参数}
          \PY{n}{cross\PYZus{}rate} \PY{o}{=} \PY{l+m+mf}{0.8} \PY{c+c1}{\PYZsh{} 杂交率}
          \PY{n}{mut\PYZus{}rate} \PY{o}{=} \PY{l+m+mf}{0.1} \PY{c+c1}{\PYZsh{} 变异率}
\end{Verbatim}


    开始遗传进化的循环操作。

    \begin{Verbatim}[commandchars=\\\{\}]
{\color{incolor}In [{\color{incolor}136}]:} \PY{c+c1}{\PYZsh{} begin}
          \PY{n}{pop\PYZus{}parent} \PY{o}{=} \PY{n}{pop}
          \PY{n}{pop\PYZus{}fit} \PY{o}{=} \PY{n}{fitness\PYZus{}pop}\PY{p}{(}\PY{n}{pop\PYZus{}parent}\PY{p}{)}
          
          \PY{n}{iter\PYZus{}num} \PY{o}{=} \PY{l+m+mi}{0}
          \PY{k}{while} \PY{n}{iter\PYZus{}num} \PY{o}{\PYZlt{}} \PY{l+m+mi}{10}\PY{p}{:}
              \PY{n}{pop\PYZus{}children}  \PY{o}{=} \PY{n}{pop\PYZus{}parent}\PY{o}{.}\PY{n}{copy}\PY{p}{(}\PY{p}{)}
              \PY{n}{prob} \PY{o}{=} \PY{n}{fitness\PYZus{}prob}\PY{p}{(}\PY{n}{pop\PYZus{}children}\PY{p}{)}  \PY{c+c1}{\PYZsh{} 计算父群适应度概率}
              \PY{n}{parent1\PYZus{}num} \PY{o}{=} \PY{n+nb}{int}\PY{p}{(}\PY{n}{cross\PYZus{}rate}\PY{o}{*}\PY{n}{num}\PY{p}{)}  \PY{c+c1}{\PYZsh{} 根据杂交概率计算杂交数量}
              \PY{c+c1}{\PYZsh{} 从个体中抽取待杂交基因编号}
              \PY{n}{cross\PYZus{}index} \PY{o}{=} \PY{n}{np}\PY{o}{.}\PY{n}{random}\PY{o}{.}\PY{n}{choice}\PY{p}{(}\PY{n}{num}\PY{p}{,} \PY{n}{parent1\PYZus{}num}\PY{p}{,} \PY{n}{replace}\PY{o}{=}\PY{k+kc}{False}\PY{p}{)} 
              \PY{k}{for} \PY{n}{i} \PY{o+ow}{in} \PY{n}{cross\PYZus{}index}\PY{p}{:} \PY{c+c1}{\PYZsh{} 待杂交个体和其它随机个体进行杂交}
                  \PY{n}{pop\PYZus{}children}\PY{p}{[}\PY{n}{i}\PY{p}{]} \PY{o}{=} \PY{n}{crossover}\PY{p}{(}\PY{n}{pop\PYZus{}children}\PY{p}{[}\PY{n}{i}\PY{p}{]}\PY{p}{,} \PY{n}{pop\PYZus{}children}\PY{p}{[}\PY{n+nb}{int}\PY{p}{(}\PY{n}{np}\PY{o}{.}\PY{n}{random}\PY{o}{.}\PY{n}{choice}\PY{p}{(}\PY{n+nb}{len}\PY{p}{(}\PY{n}{pop\PYZus{}children}\PY{p}{)}\PY{p}{,}\PY{l+m+mi}{1}\PY{p}{,}\PY{n}{p}\PY{o}{=}\PY{n}{prob}\PY{p}{)}\PY{p}{)}\PY{p}{]}\PY{p}{)}
          
              \PY{n}{mut\PYZus{}num} \PY{o}{=} \PY{n+nb}{int}\PY{p}{(}\PY{n}{mut\PYZus{}rate}\PY{o}{*}\PY{n}{num}\PY{p}{)} \PY{c+c1}{\PYZsh{} 计算待变异基因数量}
              \PY{n}{mut\PYZus{}index} \PY{o}{=} \PY{n}{np}\PY{o}{.}\PY{n}{random}\PY{o}{.}\PY{n}{choice}\PY{p}{(}\PY{n}{num}\PY{p}{,} \PY{n}{mut\PYZus{}num}\PY{p}{,}\PY{n}{replace}\PY{o}{=}\PY{k+kc}{False}\PY{p}{)}  \PY{c+c1}{\PYZsh{} 抽取待变异基因编号}
          
              \PY{k}{for} \PY{n}{i} \PY{o+ow}{in} \PY{n}{mut\PYZus{}index}\PY{p}{:}
                  \PY{n}{pop\PYZus{}children}\PY{p}{[}\PY{n}{i}\PY{p}{]} \PY{o}{=} \PY{n}{mutate}\PY{p}{(}\PY{n}{pop\PYZus{}children}\PY{p}{[}\PY{n}{i}\PY{p}{]}\PY{p}{)} 
              \PY{c+c1}{\PYZsh{} 上下双代合并排序}
              \PY{n}{pop\PYZus{}group} \PY{o}{=} \PY{n}{np}\PY{o}{.}\PY{n}{concatenate}\PY{p}{(}\PY{p}{[}\PY{n}{pop\PYZus{}children}\PY{p}{,}\PY{n}{pop\PYZus{}parent}\PY{p}{]}\PY{p}{)}
              \PY{n}{prob} \PY{o}{=} \PY{n}{fitness\PYZus{}prob}\PY{p}{(}\PY{n}{pop\PYZus{}group}\PY{p}{)}
              \PY{n}{next\PYZus{}index} \PY{o}{=} \PY{n}{np}\PY{o}{.}\PY{n}{argsort}\PY{p}{(}\PY{n}{prob}\PY{p}{)}\PY{p}{[}\PY{o}{\PYZhy{}}\PY{n}{num}\PY{p}{:}\PY{p}{]}
              \PY{n}{pop\PYZus{}parent} \PY{o}{=} \PY{n}{pop\PYZus{}group}\PY{p}{[}\PY{n}{next\PYZus{}index}\PY{p}{]} \PY{c+c1}{\PYZsh{} 优秀的变成下一代}
              \PY{n}{pop\PYZus{}fit} \PY{o}{=} \PY{n}{fitness\PYZus{}pop}\PY{p}{(}\PY{n}{pop\PYZus{}children}\PY{p}{)}
              \PY{n+nb}{print}\PY{p}{(}\PY{l+s+s2}{\PYZdq{}}\PY{l+s+s2}{第}\PY{l+s+si}{\PYZob{}n\PYZcb{}}\PY{l+s+s2}{代种群的适应度:}\PY{l+s+si}{\PYZob{}x:.3f\PYZcb{}}\PY{l+s+s2}{\PYZdq{}}\PY{o}{.}\PY{n}{format}\PY{p}{(}\PY{n}{n}\PY{o}{=}\PY{n}{iter\PYZus{}num}\PY{p}{,} \PY{n}{x}\PY{o}{=}\PY{n}{pop\PYZus{}fit}\PY{p}{)}\PY{p}{)}
              \PY{n}{iter\PYZus{}num} \PY{o}{=} \PY{n}{iter\PYZus{}num} \PY{o}{+} \PY{l+m+mi}{1}   \PY{c+c1}{\PYZsh{} 进入下一轮}
\end{Verbatim}


    \begin{Verbatim}[commandchars=\\\{\}]
第0代种群的适应度:-1.373
第1代种群的适应度:-0.389
第2代种群的适应度:-0.094
第3代种群的适应度:-0.025
第4代种群的适应度:-0.018
第5代种群的适应度:-0.005
第6代种群的适应度:-0.004
第7代种群的适应度:-0.001
第8代种群的适应度:-0.001
第9代种群的适应度:-0.001

    \end{Verbatim}

    我们总共安排了10次迭代,也就是10次进化。每次进化分为三个步骤,第一个步骤是选出高质量的个体和其它个体进行杂交,第二个步骤是选出高质量的个体进行变异,第三步是将上一代和杂交变异后的下一代合并评估,找出质量较高的个体保留,进入下一轮进化。可以观察到每一轮进化得到的种群质量是不断提高的。

    \begin{Verbatim}[commandchars=\\\{\}]
{\color{incolor}In [{\color{incolor}137}]:} \PY{n}{pop\PYZus{}children}
\end{Verbatim}


\begin{Verbatim}[commandchars=\\\{\}]
{\color{outcolor}Out[{\color{outcolor}137}]:} array([-0.03408096, -0.03408096, -0.03408096, -0.03408096, -0.03408096,
                 -0.03408096, -0.04162594, -0.03408096,  0.02211625,  0.00873233])
\end{Verbatim}
            
    \begin{Verbatim}[commandchars=\\\{\}]
{\color{incolor}In [{\color{incolor}139}]:} \PY{n}{plt}\PY{o}{.}\PY{n}{plot}\PY{p}{(}\PY{n}{x}\PY{p}{,}\PY{n}{y}\PY{p}{)}\PY{p}{;}
          \PY{n}{pop\PYZus{}fit} \PY{o}{=} \PY{p}{[}\PY{n}{func}\PY{p}{(}\PY{n}{i}\PY{p}{)} \PY{k}{for} \PY{n}{i} \PY{o+ow}{in} \PY{n}{pop\PYZus{}children}\PY{p}{]}
          \PY{n}{plt}\PY{o}{.}\PY{n}{scatter}\PY{p}{(}\PY{n}{pop\PYZus{}children}\PY{p}{,} \PY{n}{pop\PYZus{}fit}\PY{p}{,}\PY{n}{c}\PY{o}{=}\PY{l+s+s1}{\PYZsq{}}\PY{l+s+s1}{r}\PY{l+s+s1}{\PYZsq{}}\PY{p}{)}\PY{p}{;}
\end{Verbatim}


    \begin{center}
    \adjustimage{max size={0.9\linewidth}{0.9\paperheight}}{output_46_0.png}
    \end{center}
    { \hspace*{\fill} \\}
    
    现在再来观察进化后的种群,所有的X都非常靠近0,可以认为种群进化成功,找到了环境的最优点,或者说二次方函数的最小值。

    \subsubsection{11.3 练习}\label{ux7ec3ux4e60}

    使用上述的两种方法,去寻找sin函数在-10到+10之间的最小值在哪里。它的图形对应如下。

    \begin{Verbatim}[commandchars=\\\{\}]
{\color{incolor}In [{\color{incolor}145}]:} \PY{n}{x} \PY{o}{=} \PY{n}{np}\PY{o}{.}\PY{n}{linspace}\PY{p}{(}\PY{o}{\PYZhy{}}\PY{l+m+mi}{10}\PY{p}{,}\PY{l+m+mi}{10}\PY{p}{,}\PY{l+m+mi}{100}\PY{p}{)}
          \PY{n}{y} \PY{o}{=} \PY{n}{np}\PY{o}{.}\PY{n}{sin}\PY{p}{(}\PY{n}{x}\PY{p}{)}
          \PY{n}{plt}\PY{o}{.}\PY{n}{plot}\PY{p}{(}\PY{n}{x}\PY{p}{,}\PY{n}{y}\PY{p}{)}\PY{p}{;}
\end{Verbatim}


    \begin{center}
    \adjustimage{max size={0.9\linewidth}{0.9\paperheight}}{output_50_0.png}
    \end{center}
    { \hspace*{\fill} \\}
    
    提示一下,如果不知道sin函数对应的梯度函数,可以直接用sympy模块来计算,结果发现cos函数。

    \begin{Verbatim}[commandchars=\\\{\}]
{\color{incolor}In [{\color{incolor}147}]:} \PY{k+kn}{from} \PY{n+nn}{sympy} \PY{k}{import} \PY{n}{Symbol}\PY{p}{,} \PY{n}{diff}\PY{p}{,}\PY{n}{sin}
          \PY{n}{x} \PY{o}{=} \PY{n}{Symbol}\PY{p}{(}\PY{l+s+s2}{\PYZdq{}}\PY{l+s+s2}{x}\PY{l+s+s2}{\PYZdq{}}\PY{p}{)}
          \PY{n}{diff}\PY{p}{(}\PY{n}{sin}\PY{p}{(}\PY{n}{x}\PY{p}{)}\PY{p}{,}\PY{n}{x}\PY{p}{)}
\end{Verbatim}


\begin{Verbatim}[commandchars=\\\{\}]
{\color{outcolor}Out[{\color{outcolor}147}]:} cos(x)
\end{Verbatim}
            
    受篇幅所限,我们只尝试梯度下降方法,遗传算法可以读者自行尝试。

    \begin{Verbatim}[commandchars=\\\{\}]
{\color{incolor}In [{\color{incolor}1}]:} \PY{k}{def} \PY{n+nf}{min\PYZus{}gred}\PY{p}{(}\PY{n}{x\PYZus{}start}\PY{p}{,} \PY{n}{rate}\PY{p}{,} \PY{n}{num}\PY{p}{,} \PY{n}{f}\PY{p}{,}\PY{n}{g}\PY{p}{)}\PY{p}{:}
            \PY{n}{x} \PY{o}{=} \PY{n}{x\PYZus{}start}
            \PY{k}{for} \PY{n}{n} \PY{o+ow}{in} \PY{n+nb}{range}\PY{p}{(}\PY{n}{num}\PY{p}{)}\PY{p}{:}
                \PY{n}{gx} \PY{o}{=} \PY{n}{g}\PY{p}{(}\PY{n}{x}\PY{p}{)}
                \PY{n}{y} \PY{o}{=} \PY{n}{f}\PY{p}{(}\PY{n}{x}\PY{p}{)}
                \PY{n}{x} \PY{o}{=} \PY{n}{x} \PY{o}{\PYZhy{}} \PY{n}{rate}\PY{o}{*}\PY{n}{gx} \PY{c+c1}{\PYZsh{} 梯度下降}
                \PY{n+nb}{print}\PY{p}{(}\PY{l+s+s2}{\PYZdq{}}\PY{l+s+s2}{X:}\PY{l+s+si}{\PYZob{}x:.2f\PYZcb{}}\PY{l+s+s2}{, Y:}\PY{l+s+si}{\PYZob{}y:.2f\PYZcb{}}\PY{l+s+s2}{,gx:}\PY{l+s+si}{\PYZob{}gx:.2f\PYZcb{}}\PY{l+s+s2}{\PYZdq{}}\PY{o}{.}\PY{n}{format}\PY{p}{(}\PY{n}{x}\PY{o}{=}\PY{n}{x}\PY{p}{,} \PY{n}{y}\PY{o}{=}\PY{n}{y}\PY{p}{,}\PY{n}{gx}\PY{o}{=}\PY{n}{gx}\PY{p}{)}\PY{p}{)}
                \PY{k}{if} \PY{n+nb}{abs}\PY{p}{(}\PY{n}{gx}\PY{p}{)}\PY{o}{\PYZlt{}}\PY{l+m+mf}{0.0001}\PY{p}{:}
                    \PY{k}{break}
            \PY{k}{return} \PY{n}{x}     
\end{Verbatim}


    \begin{Verbatim}[commandchars=\\\{\}]
{\color{incolor}In [{\color{incolor}3}]:} \PY{n}{f} \PY{o}{=} \PY{k}{lambda} \PY{n}{x}\PY{p}{:}\PY{n}{np}\PY{o}{.}\PY{n}{sin}\PY{p}{(}\PY{n}{x}\PY{p}{)}
        \PY{n}{g} \PY{o}{=} \PY{k}{lambda} \PY{n}{x}\PY{p}{:}\PY{n}{np}\PY{o}{.}\PY{n}{cos}\PY{p}{(}\PY{n}{x}\PY{p}{)}
\end{Verbatim}


    \begin{Verbatim}[commandchars=\\\{\}]
{\color{incolor}In [{\color{incolor}9}]:} \PY{n}{min\PYZus{}gred}\PY{p}{(}\PY{l+m+mf}{0.1}\PY{p}{,}\PY{l+m+mf}{0.2}\PY{p}{,}\PY{l+m+mi}{20}\PY{p}{,}\PY{n}{f}\PY{p}{,}\PY{n}{g}\PY{p}{)}
\end{Verbatim}


    \begin{Verbatim}[commandchars=\\\{\}]
X:-0.10, Y:0.10,gx:1.00
X:-0.30, Y:-0.10,gx:1.00
X:-0.49, Y:-0.29,gx:0.96
X:-0.67, Y:-0.47,gx:0.88
X:-0.82, Y:-0.62,gx:0.79
X:-0.96, Y:-0.73,gx:0.68
X:-1.07, Y:-0.82,gx:0.57
X:-1.17, Y:-0.88,gx:0.48
X:-1.25, Y:-0.92,gx:0.39
X:-1.31, Y:-0.95,gx:0.32
X:-1.36, Y:-0.97,gx:0.26
X:-1.40, Y:-0.98,gx:0.21
X:-1.44, Y:-0.99,gx:0.17
X:-1.46, Y:-0.99,gx:0.13
X:-1.49, Y:-0.99,gx:0.11
X:-1.50, Y:-1.00,gx:0.09
X:-1.52, Y:-1.00,gx:0.07
X:-1.53, Y:-1.00,gx:0.05
X:-1.54, Y:-1.00,gx:0.04
X:-1.54, Y:-1.00,gx:0.04

    \end{Verbatim}

\begin{Verbatim}[commandchars=\\\{\}]
{\color{outcolor}Out[{\color{outcolor}9}]:} -1.5426802075384614
\end{Verbatim}
            
    可以观察到使用梯度下降方法,可以找到一个极小值,就是当x为-1.54的时候,y得到-1,在这个范围中sin函数有很多个极小值,如果需要算出其它极小值,你需要把初始值更改一下试试。

    \subsubsection{本课小结:}\label{ux672cux8bfeux5c0fux7ed3}

\begin{itemize}
\tightlist
\item
  最优化是非常关键的AI技术之一,它将是后续学习机器学习的重要概念。最优化的目标就是找到函数的极值,本课是介绍找极小值,如果是找极大值只需要将函数加个负号即可。
\item
  梯度下降法是一种局部的最优化技术,它是利用了函数的导数特性,顺着地势低处走,可以快速找到某个极小值。
\item
  遗传算法是利用自然界物竞天择的思路,每一代中选择能适应环境的个体进入下一代,它是一种可以找到全局最优的方法,不过在速度上会略慢。
\end{itemize}


    % Add a bibliography block to the postdoc
    
    
    
    \end{document}
